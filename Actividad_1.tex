\section{Actividad 1}

\noindent \textbf{a) Si las señales recibidas se suman en la antena receptora. ¿Cuál es el resultado de 
esto? Graficar. }
\bigskip

Las señales recibidas se expresan de la siguiente manera:

\[
y_1(t) = 0.9 \cdot 17 \cos(2\pi f_1 (t - \tau_1))
\]
\[
y_2(t) = 0.75 \cdot 17 \cos(2\pi f_1 (t - \tau_2))
\]

donde los retardos temporales son calculados como:

\[
\tau_1 = \frac{D_1}{c} = \frac{11000}{3 \times 10^8} = 36.67 \, \mu s
\]
\[
\tau_2 = \frac{D_2}{c} = \frac{14500}{3 \times 10^8} = 48.33 \, \mu s
\]

La señal resultante es la suma de la señal transmitida por un camino directo y de la señal reflejada \(y(t) = y_1(t) + y_2(t)\) 
\bigskip


\begin{figure}[H]
\centering
\includegraphics[width=0.8\textwidth]{parte_teorica/grafico_532kHz.png}
\caption{Señal resultante a 532 kHz.}
\end{figure}
\bigskip

En la figura 1 se observa la señal cosenoidal trasmitida en su camino directo y reflejado, la suma de las dos es la que llega a la antena receptora. Como se observa, tienen diferente amplitud y fase debido a las atenuaciones y un retardo temporal por las diferentes distancias.
\bigskip

\noindent \textbf{b) Suponer ahora que la frecuencia aumenta a 600 kHz. ¿Qué sucede? Graficar. }
\bigskip

Hay dos trayectorias:
\[
D_1 = \SI{11}{km} \qquad D_2 = \SI{14.5}{km}.
\]
\bigskip
La diferencia es:
\[
\Delta D = D_2 - D_1 = \SI{3.5}{km} = \SI{3500}{m}.
\]
\bigskip
El retardo entre ambas:
\[
\Delta \tau = \frac{\Delta D}{c} = 
\frac{3500}{3 \cdot 10^{8}}
\approx 1.1667 \times 10^{-5}\,\text{s}.
\]

\bigskip
La diferencia de fase entre ellas es:
\[
\Delta\varphi = 2\pi f \,\Delta\tau.
\]


Las dos señales quedan en fase cuando su diferencia de fase es un múltiplo entero de \(2\pi\):
\[
\Delta\varphi = 2\pi n,\qquad n=0,1,2,\dots
\]


\[
2\pi f \,\Delta\tau = 2\pi n \quad\Longrightarrow\quad f=\frac{n}{\Delta\tau}.
\]
\bigskip
Para \(\Delta\tau=1.1667\times10^{-5}\,\text{s}\):
\[
f_n=\frac{n}{1.1667\times10^{-5}}.
\]
\bigskip
Para \(n=7\):
\[
f_7=\frac{7}{1.1667\times10^{-5}}=600\,\text{kHz}.
\]
\bigskip

A \(f=600\,\text{kHz}\) la diferencia de fase es:
\[
\Delta\varphi=2\pi f \,\Delta\tau=2\pi\cdot 600000\cdot1.1667\times10^{-5}
=2\pi\cdot 7=14\pi,
\]
que es exactamente 7 ciclos completos de diferencia. Por lo tanto, las señales quedan en fase.


\begin{figure}[H]
\centering
\includegraphics[width=0.8\textwidth]{parte_teorica/grafico_600kHz.png}
\caption{Señal resultante a 600 kHz.}
\end{figure}
