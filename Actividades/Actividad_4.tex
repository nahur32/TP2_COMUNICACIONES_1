\section{Actividad 4}

\subsection*{Graficar las señales senoidales correspondientes que cumplan las siguientes condiciones:}

\subsection*{a) Señal de amplitud \(A\) aleatoria uniforme, distribuida entre \([8, 40]\).}

    \[
        x(t) = A \cos(2 \pi 10 t)
    \]

\subsection*{b) Señal de fase \(\Theta\) aleatoria uniforme, distribuida entre \([-\pi, \pi]\).}

    \[
        x(t) = 5 \cos(2 \pi 10 t + \theta)
    \]

\subsection*{c) Señal de frecuencia \(f\) aleatoria uniforme, distribuida entre \([0, 20 ]\).}

    \[
        x(t) = 5 \cos(2 \pi f t)
    \]

\subsection*{d) Señal de amplitud, fase y frecuencia aleatoria, distribuidas de igual forma que los
puntos anteriores.}

\subsection*{Además, graficar la auto‐correlación para todos los puntos anteriores. ¿Qué información se obtienen al observar la gráfica de auto‐correlación?.}